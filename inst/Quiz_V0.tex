\documentclass[12pt]{article}
%%%%%%%%%%%%%%%%%%%%%%%%%%%%%%%%%%%%%%%%%%%%%%%%%%%%%%%%%%%%%%%%%%%%%%%%%%%%%%%%%%%%%%%%%%%%%%%%%%%%%%%%%%%%%%%%%%%%%%%%%%%%%%%%%%%%%%%%%%%%%%%%%%%%%%%%%%%%%%%%%%%%%%%%%%%%%%%%%%%%%%%%%%%%%%%%%%%%%%%%%%%%%%%%%%%%%%%%%%%%%%%%%%%%%%%%%%%%%%%%%%%%%%%%%%%%
\usepackage{fancyhdr}
\usepackage{amssymb,amsmath,mathrsfs}
\usepackage{setspace}
\usepackage{float}

%TCIDATA{OutputFilter=Latex.dll}
%TCIDATA{Version=5.50.0.2953}
%TCIDATA{<META NAME="SaveForMode" CONTENT="1">}
%TCIDATA{BibliographyScheme=Manual}
%TCIDATA{LastRevised=Wednesday, June 13, 2012 11:08:07}
%TCIDATA{<META NAME="GraphicsSave" CONTENT="32">}

\setlength{\oddsidemargin}{0in}
\setlength{\textwidth}{6.5in}
\setlength{\topmargin}{-1cm}
\setlength{\textheight}{9in}
\setlength{\parskip}{5pt plus 2pt minus 3pt}
\setlength{\parindent}{0in}
\renewcommand{\baselinestretch}{1.1}
\newtheorem{definition}{Definition}
\newtheorem{theorem}{Theorem}
\newtheorem{lemma}[theorem]{Lemma}
\newtheorem{corollary}[theorem]{Corollary}
\newtheorem{claim}[theorem]{Claim}
\newtheorem{fact}[theorem]{Fact}
\lhead{\fancyplain{}{ STAT 7600}}
\rhead{\fancyplain{}{{\bf Quiz}\quad December 3, 2019}}
\newcommand{\qedsymb}{\mbox{ }~\hfill~{\rule{2mm}{2mm}}}
\newenvironment{proof}{\begin{trivlist}
\item[\hspace{\labelsep}{\bf\noindent Proof: }]
}{\qedsymb\end{trivlist}}
%\doublespacing
%\input{tcilatex}

\begin{document}


\pagestyle{fancy}

%\emph{Review for the previous lecture}
%
%\emph{Review for the previous lecture}

%\textbf{Probability theory}: sample space, sigma algebra, probability
%function
%
%\textbf{Theorem}: define a probability function on finite and countable
%infinite sample spaces
%
%\textbf{Theorem}: How to calculate probabilities of events
%\textbf{Theorem}: Bayes' Rule
%
%\textbf{Example}: how to calculate the conditional probability
%
%\textbf{Example:} how to calculate probabilities of events (using disjoint, independent, De Morgan's Law), how to calculate the distribution of a random variable

\begin{center}
\large{\textbf{QUESTIONS}}
\end{center}
%\bigskip
Show all your work to get the full credit.\\
\textbf{Question 1.}  (50 pts)
Let $c$ be a real constant and the joint pdf of $X$ and $Y$ be given as
$$f_{X,Y}(x,y)=
\begin{cases}
    c\,x y & \mbox{if } 0<x<y<1 \\
    0 & \mbox{otherwise}.
  \end{cases}
$$
(a) Find $c$ that makes $f_{X,Y}(x,y)$ a legitimate pdf.
\vspace{4cm}

(b) Are $X$ and $Y$ (stochastically or probabilistically) independent?
\vspace{3cm}

(c) Find $P(3/4 \leq Y \leq 7/8~ |~ X=1/4)$.
\vspace{4cm}

\newpage
\textbf{Question 2.} (20 pts)  If $X$ and $Y$ are two independent variables, does $E(Y|X=x)$ depend on $x$?
Explain your answer.
\vspace{5cm}

\textbf{Bonus Question:} (30 pts)
%   1) An electronic device runs until one of its two components fails. The life times (in weeks), $X_1$ and $X_2$ of these components are independent and each has Weibull pdf
%   \[f(x)=\frac{2x}{25}e^{-(x/5)^2}, ~~~0<x<\infty \]
%   Find the probability that device stops running in the first three weeks.
Let $X_1$ and $X_2$ be the two exponentially distributed independent random variables with parameter $\beta$.
Find the pdf of $Y=(X_1+X_2)/2$.
\vspace{10cm}

Some (possibly) Useful Formulas:\\
1. The pdf and MGF of $X \sim EXP(\beta)$ r.v. are
$$f(x | \beta)=\frac{1}{\beta}e^{-x/\beta}, \text{ for } 0<x<\infty,\, \beta>0
~~~\text{ and }~~~
M_{X}(t)=1/(1-\beta t) \text{ for } t<1/\beta.$$

2. The pdf and MGF of $X \sim Gamma(\alpha, \beta)$ r.v. are
$$f(x |\alpha, \beta)=\frac{1}{\Gamma(\alpha) \beta^{\alpha}} x^{\alpha-1}e^{-x/\beta}, \text{ for } 0<x<\infty,\, \beta>0,\,\alpha>0
\text{ and }M_{X}(t)=\left(\frac{1}{1-\beta t}\right)^{\alpha}.$$

%\emph{Mean}: $EX=r\frac{1-p}{p}$,
%
%
%\emph{Variance}: $VarX=r\frac{1-p}{p^{2}}$


\end{document}
